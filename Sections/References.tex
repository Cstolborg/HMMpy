\newpage
\section{References}

Ang, A. and A. Timmermann. “Regime changes and financial markets.” Annual Review of Financial Economics, vol. 4, no. 1 (2012), pp. 313–337.

Ang, A. and G. Bekaert. “International asset allocation with regime shifts.” Review of Financial Studies, vol. 15, no. 4 (2002), pp. 1137–1187

Ang, A. and G. Bekaert. “How regimes affect asset allocation.” Financial Analysts Journal, vol. 60, no. 2 (2004), pp. 86–99..

Bansal, R., M. Dahlquist, and C. R. Harvey. “Dynamic trading strategies and portfolio choice.” Working Paper 10820, National Bureau of Economic Research (2004).

Bauer, R., R. Haerden, and R. Molenaar. “Asset allocation in stable and unstable times.” Journal of Investing, vol. 13, no. 3 (2004), pp. 72–80.

Baum, L. E., Petrie, T., and Richard E. (1966). "Statistical Inference for Probabilistic Functions of Finite State Markov Chains". The Annals of Mathematical Statistics, 1 (1), 3-15

Baum, L. E., Petrie, T., Soules, G., and Weiss, N. (1970). "A maximization technique occurring in the
statistical analysis of probabilistic functions of Markov chains". Annals of Mathematical Statistics, 41, 164–171.

Bellman, R. E. “Dynamic programming and Lagrange multipliers.” Proceedings of the National Academy of Sciences, vol. 42, no. 10 (1956), pp. 767–769.

Bemporad, A., Breschi, V., Piga, D., and Boyd, S. P. "Fitting jump models". Automatica, 96:11–21, 2018.

Boyd, S., E. Busseti, S. Diamond, R. N. Kahn, K. Koh, P. Nystrup, and J. Speth. “Multi-period trading via convex optimization.” Foundations and Trends in Optimization, vol. 3, no. 1 (2017), pp. 1–76.

Boyd, S., M. T. Mueller, B. O’Donoghue, and Y. Wang. “Performance bounds and suboptimal policies for multi-period investment.” Foundations and Trends in Optimization, vol. 1, no. 1 (2014), pp. 1–72.

Boyd, S. and L. Vandenberghe. Convex Optimization. Cambridge University Press: New York (2004).

Brinson, G. P., Hood, L. R., Beebower, G. L., 1986. Determinants of portfolio performance. Financial Analysts Journal 42 (4), 39–44.

Bulla, J., 2011. Hidden Markov models with t components. Increased persistence and other aspects. Quantitative Finance 11 (3), 459–475.

Bulla, J., Bulla, I., 2006. Stylized facts of financial time series and hidden semi- Markov models. Computational Statistics and Data Analysis 51, 2192–2209.

Bulla, J., S. Mergner, I. Bulla, A. Sesboüé, and C. Chesneau. “Markov-switching asset allocation: Do profitable strategies exist?” Journal of Asset Management, vol. 12, no. 5 (2011), pp. 310–321.

Campbell, J. Y. “Asset prices, consumption, and the business cycle.” In Handbook of Macroeconomics, edited by J. B. Taylor and M. Woodford, vol. 1C, chap. 19. Elsevier: Amsterdam (1999), pp. 1231–1303.

Cochrane, J. H. “Financial markets and the real economy.” Foundations and Trends in Finance, vol. 1, no. 1 (2005), pp. 1–101.

Cont, R. "Empirical properties of asset returns: stylized facts and statistical issues". Quantitative Finance 1, 223-236 (2001).

Dempster, A. P., Laird, N. M., and Rubin D. B. (1977). "Maximum Likelihood from Incomplete Data via the EM Algorithm". Journal of the Royal Statistical Society, 39 (1), 1-38.

Fama, E., French, F. and Kenneth, R. (1989). Business conditions and expected returns on stocks and bonds, Journal of Financial Economics, 25, 23-49.

Fornaciari, M. and C. Grillenzoni. “Evaluation of on-line trading systems: Markov-switching vs time-varying parameter models.” Decision Support Systems, vol. 93 (2017), pp. 51–61.

Gârleanu, N. and L. H. Pedersen. “Dynamic trading with predictable returns and
transaction costs.” Journal of Finance, vol. 68, no. 6 (2013), pp. 2309–2340.

Granger, C. W. J. and Z. Ding. “Stylized facts on the temporal and distributional properties of daily data from speculative markets.” Unpublished paper, Department of Economics, University of California, San Diego (1995b).

Géron, A. (2019). Hands-On Machine Learning with Scikit-Learn, Keras, and TensorFlow: Concepts, Tools, and Techniques to Build Intelligent Systems. O’Reilly Media.

Guidolin, M. and A. Timmermann. “Asset allocation under multivariate regime switching.” Journal of Economic Dynamics and Control, vol. 31, no. 11 (2007), pp. 3503–3544.

Goltz, F., L. Martellini, and K. D. Simsek. “Optimal static allocation decisions
in the presence of portfolio insurance.” Journal of Investment Management,
vol. 6, no. 2 (2008), pp. 37–56.

Grossman, S. J. and Z. Zhou. “Optimal investment strategies for controlling
drawdowns.” Mathematical Finance, vol. 3, no. 3 (1993), pp. 241–276.

Hamilton, J. D., 1989. A new approach to the economic analysis of nonstationary time series and the business cycle. Econometrica 57 (2), 357–384.

Hesselholt, Mads and Teis Knuthsen, (2009). Investeringsstrategi på tværs af konjunkturen, Nykredit Asset Management.

Ibbotson, R. G. and P. D. Kaplan. “Does asset allocation policy explain 40, 90, or 100 percent of performance?” Financial Analysts Journal, vol. 56, no. 1 (2000), pp. 26–33.

Kritzman, M. and Y. Li. “Skulls, financial turbulence, and risk management.” Financial Analysts Journal, vol. 66, no. 5 (2010), pp. 30–41.

Kritzman, M., S. Page, and D. Turkington. “Regime shifts: Implications for dynamic strategies.” Financial Analysts Journal, vol. 68, no. 3 (2012), pp. 22–39.

Kolm, P., R. Tütüncü, and F. Fabozzi. “60 years of portfolio optimization: Practical challenges and current trends.” European Journal of Operational Research, vol. 234, no. 2 (2014), pp. 356–371.

Mandelbrot, B. “The variation of certain speculative prices.” Journal of Business, vol. 36, no. 4 (1963), pp. 394–419.

Markowitz, H. “Portfolio selection.” Journal of Finance, vol. 7, no. 1 (1952), pp. 77–91.

Merton, R. C. “Lifetime portfolio selection under uncertainty: The continuoustime case.” Review of Economics and Statistics, vol. 51, no. 3 (1969), pp. 247–257

Merton, R. C. “Theory of rational option pricing.” Bell Journal of Economics
and Management Science, vol. 4, no. 1 (1973), pp. 141–183.

Mossin, J. “Optimal multiperiod portfolio policies.” Journal of Business, vol. 41,
no. 2 (1968), pp. 215–229.

Munk, C (2019). Financial Markets and Investments. \textit{Copenhagen Business School}.

Nystrup, P., Lindström, E., and Madsen, H. "Learning hidden Markov models with persistent states by penalizing jumps". Expert Systems with Applications, 150:113307, 2020b.

Thompson, The Stylised Facts of Stock Price Movements. NZREF 1 (2011).

Pedersen, L. H. “When everyone runs for the exit.” International Journal of Central Banking, vol. 5, no. 4 (2009), pp. 177–199.

Quandt, R. E., (1958). The Estimation of the Parameters of a Linear Regression System Obeying Two Separate Regimes. Journal of the American Statistical Association, 53 (284), 873-880.

Quandt, R. E., (1972). A New Approach to Estimating Switching Regressions. Journal of the American
Statistical Association, 67 (338), 306-310.

Rabiner, L. R. (1989). A tutorial on Hidden Markov Models and selected applications in speech recognition, Proceedings of the IEEE, 77, 257–286.

Rogers, L., \& Zhang, L. (2011). Understanding Asset Returns. Mathematics and Financial Economics,
5(2), 101-119.

Ross, G. J., D. K. Tasoulis, and N. M. Adams. “Nonparametric monitoring of data streams for changes in location and scale.” Technometrics, vol. 53, no. 4 (2011), pp. 379–389.

Rydén, T., T. Teräsvirta, and S. Åsbrink. “Stylized facts of daily return series and the hidden Markov model.” Journal of Applied Econometrics, vol. 13, no. 3 (1998), pp. 217–244.

Sheikh, A. Z. and J. Sun. “Regime change: Implications of macroeconomic shifts on asset class and portfolio performance.” Journal of Investing, vol. 21, no. 3 (2012), pp. 36–54.

Siegel, J. J. “Does it pay stock investors to forecast the business cycle?” Journal of Portfolio Management, vol. 18, no. 1 (1991), pp. 27–34.

Zakamulin, V. “Dynamic asset allocation strategies based on unexpected volatility.” Journal of Alternative Investments, vol. 16, no. 4 (2014), pp. 37–50.

Zheng, K., Li, Y., and Xu, W. "Regime switching model estimation: spectral clustering hidden Markov model". Annals of Operations Research, pages 1–23, 2019.

Zucchini, W. and I. L. MacDonald. Hidden Markov Models for Time Series: An Introduction Using R. Chapman \& Hall: London, 2nd ed. (2009).








