\section{Motivation and introduction}
\label{section: introduction}


\begin{itemize}
    \item Generelt intro til DAA, regimed-based asset allocation og brugen af HMMs til dette - ca. 1 side.
    \item Thesis goals. I dette afsnit beskriver vi målet ved thesen. Den generelle opbygning gennemgås og der gives delkonklusioner fra hver subsektion - ca 1-1.5 side.
    \item Evt. yderligere beskrivelser af HMMs litterature reivew, valg af time frequency etc etc.
\end{itemize}

\begin{enumerate}
    \item Motiver emnet. Kort gennemgang af nogle af de studier vi lægger os tæt op af, hvilket interessante findings de havde og hvordan man kan bygge videre på dem - dvs. mangler - Bulla, Ryden, Nystrup - 1 side.
    \item Thesis statement. Fokus på sektion 5-7. Beskriv først hvorfor hvert af de 3 emner er interessante og hvordan de spiller sammen (1 paragraf) og gennemgå dem derefter en ad gangen (1 paragraf per emne).
    \item Litterature review
\end{enumerate}

The behavior of financial markets has been, and continues to be, subject to abrupt changes. As such, the GFC of 2008 showed that means, volatility and correlation patterns in stock returns changed considerable in the years that followed and the investors' risk appetite decreased across asset classes (Uhlenbrock, 2009). Furthermore, the GFC demonstrated that diversification is insufficient to mitigate large drawdowns in periods where it is needed the most since the correlation between risky assets tend to strengthen during periods of high market volatility (Pedersen 2009). 

Despite this, asset allocation, defined as the choice of weights between major asset classes, has proven and continuous to be the most important determinant regarding portfolio performance (Brinson et al. 1986) and (Ibbotson \& Kappland, 2000). However, the time-varying nature of investment opportunities suggests that the weights underlying the portfolio allocation should be adjusted as new information arise (Bansal et. 2004). One of the methodologies that most recently has been proposed in order to conduct optimal asset allocation involves the use of regime-switching HMMs which identifies a number of economic regimes after which a capital allocation is made based on the inferred regime (Nystrup, 2014) and (Nystrup, 2017). Even though the studies showed that performing capital allocation on the basis of inferred regimes yields supperior performance in terms of risk-adjusted returns when compared to "plain-vanilla" portfolio strategies such as an equally-weighted portfolio, the studies neglected some crucial considerations that are paramount in quantitative finance. Firstly, both of the mentioned studies have a strong focus on reproducing the long memory of financial returns, however, neither manages to reproduce the set of stylized facts introduced by Granger \& Ding (1995b). As such, one of the key motivating aspects of this thesis is to expand on the work by Nystrup (2014, 2017) in an attempt to train and fit HMMs that are capable of reproducing the set of stylized facts defined by Granger \& Ding (1995b). When doing so, the thesis will draw direct comparisons to the studies conducted by Rydén et. al (1998) and Bulla (2011) who fitted conditional Gaussian and t-distributed HMMs respectively, in an attempt to reproduce the stylized facts defined by Granger \& Ding (1995b). 

Even though both Rydén et al. (1998) and Bulla (2011) somewhat managed to reproduce the set of stylized facts some properties had to be relaxed hence the the thesis will attempt to obtain an improved fit. Furthermore, both Rydén et al. (1998) and Bulla (2011) neglected to examine whether profitable trading strategies could be implemented based on the models that they derived. As such, the essence of the motivation underlying this thesis is to combine and expand on the work by Nystrup (2014, 2017) as well as Rýden et al (1998) and Bulla (2011) by fitting conditional Gaussian HMMs followed by a detailed analysis of whether the models are able to reproduce the set of stylized facts. Assuming this is the case, the models will be used in a portfolio exercise to test whether profitable trading strategies exist. Despite the fact that Nystrup (2014) presented and tested a variety of portfolio strategies rooted in the mean-variance framework, the overall idea involved re-calibrating the portfolio weights based on the inferred regime. As such, the optimal portfolios were trained and defined in-sample and then used out-of-sample based on the inferred regimes. 

An alternative approach to simply switching between a static decision rule is to dynamically optimize the portfolio while adjusting for transactions costs, risk aversion, maximum drawdown as well as a variety of other constraints. This methodology is known as model predictive control (MPC). This methodology was presented by Nystrup (2017) who expanded on the original work of Boyd et al. (2014). As such, rather than relying on a static re-calibration to a set of optimal portfolios defined in-sample, the MPC will rely on the HMMs to make forecasts of the asset returns on a daily basis. The key motivation behind selecting this portfolio framework is rooted in the shortcomings of the mean-variance approach, evident by the financial crisis which saw even diversified investor loose capital (Pedersen, 2009). Secondly, the MPC provides a more realistic approach to portfolio management as it can account for any constraint or requirements that a specific investor could have. Lastly, the framework is more dynamic in nature, since it involves making a decision whether or not to change capital allocation on a daily basis rather than simply re-calibrating to a set of pre-defined optimal portfolios.

Even though the above would yield a comprehensive analysis, it is important to note that all of the aforementioned studies only included HMMs estimated through the MLE approach. It was only in a later study by Nystrup (2020), that he expanded on the work of Bemporad et al. (2018) by attempting to fit HMMs through the jump framework. However, the study only focused on actually estimating the HMMs hence it was not examined whether the jump models could reproduce the set of stylized facts and no trading strategies were tested based on the estimated models. As such, the study by Nystrup (2020) provides a gap in the literature of quantitative finance since the study proved that HMMs can be estimated by a jump framework, however, any additional analysis was neglected. As such, the final key motivation underlying this thesis is to expand and contribute to the literature within quantitative finance and portfolio theory by analysing whether the jump model can reproduce the set of stylized facts put forward by Granger \& Ding (1995b) which is then followed by a portfolio exercise to examine whether one can build profitable trading strategies based on the forecastsed asset returns of HMMs estimated through a jump framework. Naturally, the study will continuously focus on the differences between the \mle and \jump approach, both in terms of model convergence, how well each methodology is able to reproduce the stylized facts and in terms of differences in the obtained risk-adjusted returns. Having set the stage through motivating the subsequent analysis, the following section will entail an outline of the thesis in order to further specify the different steps of the analysis.

% Key elementer i motivationen.
% Nystrup sammenligner ikke direkte til Granger & Ding.
% G&D samt Ryden tester ikke trading-strategier.
% Bruger kun MLE og ikke JUMP til at reproducere long memory.
% Selvom han i senere studier bruger JUMP sammenligner han ikke til G&D.


\subsection{Thesis outline}
Based on the introduction above it is clear that the motivation and overall purpose of this thesis is rooted in an objective to compare the performance of a regime-based asset allocation strategy to that of equally-weighted as well as static all-weather portfolios. However, before one can compare the results through a portfolio exercise, some additional analysis are needed. As such, the thesis will originate by introducing the general notation and assumptions regarding HMMs which will be followed by en estimation section highlighting the mathematics of two distinct approaches known as \mle and \jump estimation. Having provided an overview of the estimation procedures the thesis will proceed by conducting a simulation study designed to test how the different parameters of the models converge to the true values when simulated from correctly and incorrectly specified distributions as well as from a variety of sample lengths. This is done in order to validate or disqualify the models and determine whether there is a substantial difference between the estimation procedures in terms of how likely the models are to be subject to large estimation errors. Furthermore, the section will provide an in-depth analysis of parameter tuning, particularly in regards to the jump model. 

Once the simulation studies have proven sufficient, the thesis will introduce the financial data that will be used in terms of the actual model estimation. As such, by having both the estimation procedures and the financial data in place, the analysis will move on to estimate the models and examine whether they are able to reproduce the stylized facts introduced by Granger \& Ding (1995b) in which a direct comparison will be drawn to the studies conducted by Rydén et al. (1998) and Bulla (2011). This is one of the crucial parts of the thesis because if the models fail to reproduce the stylized facts, it becomes needless to perform a portfolio exercise. However, assuming that the estimated models are able to sufficiently reproduce the stylized facts, the thesis will conclude by conducting a portfolio exercise in which the overall objective of the thesis is uncovered, namely whether RBAA combined with MPC results in supperior risk-adjusted returns compared to an equally weighted portfolio as well as a static all-weather portfolio strategy, thereby answering the question if profitable investment strategies exist. Having outlined the motivation as well as how the analysis will progress, the subsequent section will introduce and review the literature of a variety of key topics associated with HMMs, the stylized facts of financial returns, dynamic asset allocation and portfolio optimization as well as how to properly determine the time-horizon in the data used for model estimation. 

%Python is used with regards to modeling and forecasting. The setup of the code and the associated file-management is controlled through an online Github and the PyCharm project-management software. For optimization, Python and the library CVXPY (Diamond and Boyd 2016) is used.

\textbf{Ting der skal med i Litterature review / efterfølgende afsnit. Alle større valg i opgaven skal outlines her. Dvs. der skal forklares hvorfor vi vælger at bruge 2-stats modeller, hvorfor de er normale, hvorfor vi ser på daglige returns osv. osv. Forlæng gerne nedenstående liste.}
\begin{enumerate}
    \item Generel litteratur om regime-shifts. Skal også dække andre modeller end HMM, indledningsvist, og skal så zoome ind på HMMs. Den her del skal indeholde en forklaring på valget af HMMs og derefter valget af 2-stats modeller og hvorfor 3-stats modeller ikke benyttes. Den her del kunne evt. tage primær henvising til kilder som Bulla.
    \item Review af porteføljeoptimering i aktivallokering. Alternative metoder og hvorfor MPC virker interessant at forfølge.
    \item Tidshorisont, daglig frekvens og andre mindre ting kan afslutningsvist kort forklares.
\end{enumerate}
    
\subsection{Regime shifts}
During the last two decades regime-switching models have gained substantial interest from researcher within the field of macroeconomics and quantitative finance. The regime-switching models that the researchers are utilising are part of a group of models broadly referred to as Markov-switching models. Markov-switching models are time series models that contain a hidden variable generated by an unobserved Markov process and the popular hidden Markov model is a subgroup of the Markov-switching models. The introduction of such time-varying parameter models, used to capitalize on the time-varying risk premiums as described by Fama and French (1989), dates back to Quandt (1958) who pioneered by introducing an approach that would estimate a linear regression system with two regimes. In a later study the original idea of Quandt (1958) was further expanded by Goldfeld \& Quandt (1973) in which they introduced Markov-switching regressions. Following this, the first to consider the impact of regime changes on asset allocation was Ang \& Bekaert (2002). More precisely, the authors modelled monthly returns from 1970 to 1997 as a multivariate regime-switching process with two states. This was later followed by a study by Guidolin and Timmerman (2007) who estimated a four-state Markov-switching auto-regressive model. The authors found that the optimal asset allocation varied considerably across time and the allocation towards stocks was only increasing monotonically in one of the four regimes. Furthermore, Guidolin \& Timmerman found that the persistence of the regimes were rather unstable across time, indicating a high degree of switching behavior. This finding indicates that regime-switching models can have too many potential economic regimes to choose from and it serves as one of the reasons why this thesis operates with two regimes rather than three or more. Yet, Guidolin \& Timmerman (2007) confirmed the importance of modelling portfolios based on inferred regimes as they considerably outperformed a naive investment strategy out of sample. 

In addtion, Rydén et al. (1998) were among the first to consider whether HMMs, estimated on daily asset returns, can reproduce the set of stylized facts put forward by Granger \& Ding (1995b). As such, the authors fitted two-state HMMs with conditional Gaussian distributions in which they achieved staggering results since only a few of the properties underlying the stylized facts had to be somewhat relaxed. However, even though the models were able to somewhat reproduce the long memory of financial returns, which is one of the most important properties, a perfect fit was not achieved. Despite this, the study conducted by Rydén et al. (1998) has been regarded as one of the key turning points in terms of further expanding the use of HMMs within finance and economics. Following this, Bulla et al. (2011) fitted a two-state HMM to daily stock return series. By doing so, the authors achieved a must higher degree of regime persistence when compared to the results of Guidolin \& Timmerman (2007). The study was subsequently followed by an additional study by Bulla (2011) who attempted to fit two-state HMMs with a conditional t-distribution in order to reproduce the stylized facts put forward by Granger \& Ding (1995b). The study showed particularly promising results as only a few of the properties underlying the stylized facts had to be somewhat relaxed. As such, the recent studies further solidifies the appropriateness of relying on HMMs when modelling asset returns. A similar study was later repeated by Nystrup (2014) who obtained comparable results to those of Bulla et al. (2011) and Bulla (2011). The findings by Bulla et al. (2011), Bulla (2011) and Nystrup (2014) serve as the other key reason as to why this thesis utilizes a two-state HMM in the estimation procedure. 

The theory of regime-switching models and HMMs has most recently been expanded by Nystrup et al.(2020b) who combined the jump framework of Bemporad et al. (2019) with the temporal features used by Zheng et al. (2019) in order to train HMMs. The inclusion of a jump-penalizer provides influence over the transition rates in the predicted state sequence and the researchers found that the method performed favourably over traditional HMM optimization methodologies such as maximum likelihood and spectral clustering. In conclusion, the previous research have shown that when coupling financial data with regime-switching models like HMMs one can establish a link to the general business cycle. In addition, many of the aforementioned studies achieved promising results by fitting a two-state conditional Gaussian HMM on a daily return series. As such, the thesis will utilize a similar approach rooted in the promising results but also with regards to the possibility of making direct comparisons of the obtained results to existing literature. 

\subsection{HMMs and stylized facts}
It is well known within quantitative finance that a Gaussian distribution provides a poor fit to most financial returns. This is evident as the Gaussian distribution fails to capture the fat tails exhibited by returns as well as the volatility clustering. However, mixture distributions provide a better fit since they are able to reproduce the leptokurtosis and skewness prevalent in financial return series (Cont, 2001). In order to capture both the distributional and temporal properties of financial returns, an extension to the traditional Markov switching mixture, known as HMMs have been particular promising (Bulla, 2011). The model was first introduced to the area of financial economics by Hamilton (1989) and it has since gained stronghold in quantitative finance. Furthermore, one of the key aspects with HMMs as that the inferred states can directly be linked to the different phases in the economic cycle (Ang \& Timmermann, 2012). Therefore, the fact that the model allows for interpretability of the states combined with its ability to reproduce the stylized facts makes it highly popular. 

After the introduction by Hamilton in 1989 it was shown by Rydén et al. (1998) that HMMs comprise the ability to reproduce most of the stylized facts put forward by Granger \& Ding (1995b). However, HMMs fall short when attempting to replicate the slowly decaying autocorrelation function of absolute daily returns, which is also referred to as volatility clustering (Cont, 2001). In an attempt to improve the fit of HMMs as well as their ability to reproduce the long memory property of financial returns, a variety of papers have followed and extended the findings related to Gaussian HMMs by Rydén et al. (1998). As such, Bulla \& Bulla (2006) found that the HMMs insufficient ability to reproduce the long memory property can be explained by the implicit assumption of geometrically distributed sojourn times in a particular hidden state. To circumvent this limitation, Bulla \& Bulla (2006) considered hidden semi Markov models (HSMM), which make it possible to explicitly model the sojourn time distribution for each state. Upon completion of their research, Bulla \& Bulla (2006) found that HSMMs could be better at reproducing the long term memory of daily financial returns, however, they failed to specify an optimal sojourn time distribution. Bulla (2011) further expanded on the earlier research by showing that HMMs with t-distributed components reproduce a broad array of the stylized facts as well or better than Gaussian HMMs. 


\subsection{Dynamic asset allocation and portfolio optimization}
Asset allocation encompass the decision of how to allocate and distribute capital among a set of major asset classes and although the behavior of different asset classes are subject to variation across shifting economic regimes, no individual asset class dominates across all regimes. As such, rather than adopting to macroeconomic shifts, SAA investors seek to develop static "all-weather portfolios". However, if economic conditions can be modelled with high reliability and these findings are persistent, then DAA and RBAA should add value when compared to SAA strategies. As such, the purpose of DAA and RBAA is to exploit favourable economic conditions while mitigating draw-downs and losses in high volatility markets. This means that the objective of DAA and RBAA is not to forecast and predict economic regimes but rather to identify the current regime and build investment strategies around this. As such, DAA and RBAA are much more restrictive than SAA regarding the investment opportunity set, since illiquid assets such as real estate, private equity and infrastructure are difficult to dynamically optimize (Nystrup, 2014).

It follows from Zakamulin (2014) that the DAA and RBAA framework can be split into a rule and optimization-based methodology. The rule-based approach can be described as defining a set of optimal portfolios, for each potential economic regime, and these portfolios are then traded based on the estimated regime from a HMM. However, even though one can optimize the portfolio weights in-sample, doing so does not guarantee an optimal decision rule for the problem at hand (Fornaciari \& Grillenzoni, 2017). As a natural consequence, a larger number of different specifications might have to be tried, before a decision rule with satisfactory performance can be uncovered. Yet, when testing a high degree of different specifications in-sample one increases the risk of overfitting and thus inferior out-of-sample performance. Further, it can be argued that a static decision rule is hardly optimal when the underlying HMM used for regime inference is time varying, as in Bulla et al. (2011).

In later study by Nystrup (2017) the use of model predictive control (MPC) was introduced to circumvent the issues in terms of rebalancing to a static set of predefined in-sample weights. The idea, originally presented by Boyd (2014), involves optimizing the portfolio on a daily basis, based on financial return forecasts, while taking transaction and holding costs as well as risk aversion into account. As such, it appears that MPC makes for a promising alternative to the traditional static all-weather decision rule which used to dominate the literature. As such, this serves as one of the key reasons as to why this thesis relies on the MPC framework in the portfolio exercise, because when the HMMs are updated on a daily basis it makes perfect sense to check whether the optimal portfolio allocation has changed as well and whether it is economically feasible to update the allocation if such a change has occurred. As such, the HMMs and the MPC framework is interconnected since they rely on the same data frequency in their estimation procedure. Lastly, the MPC framework will be utilized due to the fact that it allows for straight-forward implementation of different types of costs and constraints that are crucial for portfolio management. Therefore, the methodology build upon the portfolio theory put forward by Markowitz (1952) highlighting that the objective is to uncover an optimal risk-return trade-off. Even though the mean-variance criterion is the most common approach in terms of portfolio selection (Kolm et al. 2014), the MPC methodology allows for alternative risk measures such as maximum drawdown and expected shortfall. As such, the thesis will focus on implementation of the MPC approach rather than simply switching between a predefined set of portfolios based on the regime sequence.  


\subsection*{Time horizon and frequency of data}
\label{subsection: Data frequency}

When utilising financial market data to uncover economic regime changes, a natural questions arises in terms of which data frequency to use. As described, the objective of DAA is to rebalance the portfolios once a regime shifts has occurred, hence if the regime detection relies on too infrequent data, there is a high probability that several regime shifts will remain hidden. As such, data frequencies longer than a month is not considered. Furthermore, broad macroeconomic data is widely available, however, the data is typically characterised by a data frequency of months or even quarters. Therefore, the data frequency of macroeconomic variables propose a challenge, as historic events has shown that economic regime changes can happen swiftly, evident by the recent COVID-19 recession. As such, monthly data compared to e.g. daily data, greatly increases the risk of slow and insufficient detection of economic regime shifts.
 
Despite of the reasoning just outlined, it should be noted that the use of daily data presents some challenges as well. This is due to the fact that daily returns contain a lot of noise and extreme observations, which are evened out on a monthly basis. Consequently, long-horizon returns tend to be more closely approximated by a Gaussian distribution compared to returns for shorter time horizons (Campbell et al. 1997). As such, short-term data frequencies complicates the modeling significantly, which can lead to sporadic predictions and less persistence in the uncovered economic states. In addition, the use of daily return frequencies makes the aforementioned link to macroeconomic data more difficult to justify, and therefore it becomes challenging, from a macroeconomic perspective, to argue that economic regime shifts occur since there is no tangible macroeconomic evidence to support this before the data gets collected and released. Yet, despite the complications associated with short-term data frequencies, the arguments for considering daily as opposed to monthly data frequencies are compelling. 

Furthermore, despite the aforementioned issues with sporadic predictions due to daily data frequencies, Bulla et al. (2011) argued that by relying on daily data it becomes feasible to apply filters that increase confidence whenever a regime change has been detected. As such, research cements the possible of implementing a waiting scheme, in which the portfolio manager would wait several days, when a regime shifts has been detected, before changing the portfolio allocation, thereby minimizing the risk of re-allocating capital based on a wrong signal. In addition, the use of MPC makes daily data frequencies more justified, since it is not possible to make better return predictions for the assets than their long-term average. As such, looking only a limited number of days into the future is not just an approximation necessary to make the optimization problem computationally feasible, it is also reasonable due to the nature of financial returns.

Bulla (2011) further argued that the use of daily data frequencies increases the amount of data available for markets characterised by a short lifetime, however, many financial indices are old and for these older indices monthly data is more available than daily data. Unsupervised machine learning models, such as HMMs, require large amounts of data to be trained in order to achieve a desired and comfortable level of predictability and persistence, hence the usage of these heavy data-driven models serves as an argument for relying on monthly data as opposed to daily data. Yet, this thesis favors early detection, due to the reasoning outlined above. As a result, daily data, will be used in the analysis. Additionally, the optimal time horizon used for training the model is debatable, however, it should at least span the time required for a financial cycle to unfold in order to include the performance of DAA strategies in several economic environments. 

