\section{Introduction}
\label{section: introduction}

The behavior of financial markets have been, and continue to be, subject to abrupt changes. Even though some behavioural changes are transitory, their impact often persists for several periods. As such, the global financial crisis (GFC) of 2007-2008 showed that means, volatility and correlation patterns in stock returns as well as investors' risk appetite changed significantly and continued to shift dramatically across the financial markets for the years to follow. Furthermore, the GFC demonstrated that diversification is insufficient to mitigate large drawdowns in periods where it is needed the most since the correlation between risky assets tend to strengthen during periods with high uncertainty (Pedersen 2009). 

Despite this, asset allocation, defined as the choice between major asset classes, has proven to be the most important determinant regarding portfolio performance (Brinson et al. 1986, Ibbotson \& Kappland 2000). As such, time-varying investment opportunities suggest that the weights underlying the portfolio allocation should be adjusted as new information arises (Bansal et. 2004). However, even though the performance of different asset classes is strongly correlated to general economic conditions, investors utilizing strategic asset allocation (SAA), rely on optimization across a range of economic regimes in which the portfolios are re-calibrated and re-balanced at a specified time-scale, e.g. annually (Sheikh \& Sun, 2012).

A universal and macroeconomic way to identify regime changes is through Composite Leading Indicators (CLI), which provide signals of turning points in the business cycle by capturing fluctuation of economic activity around its long term level (Hesselholt Knuthsen, 2009). Yet, these indicators tend to lag behind the economic activity due to the delay in how economic data is released as this can vary from monthly to quarterly data depending on the economic figure. Therefore, there is a high risk that the regime change has already happened, when the data becomes available for analysis (Nystrup, 2014). As such, regime-switching models, based on daily financial returns series, become reliable as changes to regimes can be identified almost instantly due to how new information is continuously priced into financial assets. The following section will provide a review of developments within dynamic asset allocation (DAA), the importance of financial models capturing the stylized facts as well as how the use of Markov-switching mixture models can capture the time-varying aspect of financial returns. 

\subsection*{Regime shifts}
Regime shifts are prevalent across financial markets as well as in the behavior of macroeconomic variables. Further, regime shifts are typically either recurring, illustrated through recession and expansion, or permanent in which they take the form as structural breaks in the economy (Ang \& Timmermann, 2012). In relation to this Campell (1999) and Cochrane (2005) observed that regimes present in financial markets are related to the general economic business cycle. However, the link between regimes and the business cycle is highly complex and not easy to uncover with traditional models, and due to the lagged nature of economic data, the link has historically been difficult to exploit in terms of investment purposes. As such, the focus for this thesis will be on available financial market data rather than trying to establish a link to the business cycle. Under the assumption that financial markets are efficient, the outlook for the future economic activity should be captured in the development of asset prices (Siegel, 1991). Given this assumptions it is striking that most studies regarding DAA consider monthly rather than daily return series. A monthly data frequency might be viable for the static rule-based nature of SAA but not for DAA, since even though it is not expected that a trade will occur each day, the option daily adjusting of the portfolio should be present (Nystrup, 2017). 

The first to consider the impact of shifting regimes on asset allocation was Ang \& Bekaert (2002). As such, they found that regime-based asset allocation (RBAA) contains the potential to reduce drawdown and increase profit over simple rule-based rebalancing to static weights. Several studies have followed the original work of Ang \& Bekaert (2002), including Ang \& Bekaert (2004), Bauer et al.
(2004), Guidolin and Timmermann (2007), Bulla et al. (2011) and Kritzman et al. (2012). The majority of the listed research considered DAA across assets including stocks, bonds as well as a risk-free asset, however, the proposed solutions often lead to allocation changes that exceeds the investment mandate of what most investors are able to implement. Furthermore, the focus on stocks is plausible since the predominant portion of portfolio risk can be attributed to stock market risk. 

It is important to underline that regime shifts are an abstraction of reality that aims to capture and provide a tangible association to changes in economic variable. As such, models that are specified with a fixed number of recurring regimes, which will be discussed further below, are just one way to model and capture these economic changes see Ross et al. (2011). 


\subsection*{Stylized facts and Markov switching mixtures}
A Gaussian distribution provides a poor fit to most financial returns. This is evident as the Gaussian distribution fails to capture the fat tails exhibited by returns as well as volatility clustering. As such, mixture distributions provide a better fit since they are able to reproduce the leptokurtosis and skewness prevalent in financial return series (Cont, 2001). In order to capture both the distributional and temporal of financial returns, an extension to the traditional Markov switching mixture, known as the hidden Markov model (HMM), is applied. The model was first introduced to the area of financial economics by Hamilton (1989) and it has since gained stronghold in quantitative finance. Within HMMs, the distribution that generates an observation depends on the state of an unobserved Markov chain (Zucchini \& MacDonald, 2009). As such, HMMs can be characterised as "black-box models", however, the inferred states can directly be linked to the different phases in the economic cycle (Ang \& Timmermann, 2012). Therefore, the fact that the model allows for interpretability of the states combined with its ability to reproduce the stylized facts makes it highly popular. 

After the introduction by Hamilton in 1989 it was shown by Rydén et al. (1998) that HMMs comprise the ability to reproduce most of the stylized facts put forward by Granger \& Ding (1995a,b). However, HMMs fall short when attempting to replicate the slowly decaying autocorrelation function of absolute daily returns, which is also referred to as volatility clustering (Cont, 2001). Mandelbrot (1963) was among the first to uncover that the volatility of asset prices forms clusters, due to the fact that large prices moves tend to be followed by subsequent large price moves, however, not necessarily in the same direction. As shown by fig \textbf{(REF!!)} daily returns do not exhibit a long memory property, however, their absolute values do. 

%######### Indsæt figur af ACF og ABS(ACF) function for illustration,

In an attempt to improve the fit of hidden markov models, and thus improve its ability to reproduce the long memory property of financial markets, a variety of papers have followed and extended the findings related to Gaussian HMMs by Rydén et al. (1998). As such, Bulla \& Bulla (2006) found that the HMMs insufficient ability to reproduce the long memory property can be explained by the implicit assumption of geometrically distributed sojourn times in a particular hidden state. To circumvent this limitation, Bulla \& Bulla (2006) considered hidden semi Markov models (HSMM), which makes it possible to explicitly model the sojourn time distribution for each state. Upon completion of their research, Bulla \& Bulla found that HSMMs are better at reproducing the long term memory of daily financial returns, however, they failed to specify the most appropriate sojourn time distribution. 

Bulla (2011) expanded on the earlier research by showing that HMMs with t-distributed components reproduce a broad array of the stylized facts as well or better than Gaussian HMMs. Additionally, Bulla (2011) showed that HMMs with three states provided a better fit than HMMs with two states when calibrated against daily returns of the S\&P 500 index from 1928 to 2007. 

The theory of regime-switching models and HMMs has most recently been expanded by, Nystrup et al.(2020b) who combined the jump framework of Bemporad et al. (2019) with the temporal features used by Zheng et al. (2019) in order to learn HMMs. The inclusion of a jump-penalizer provides influence over the transition rates in the predicted state sequence, and the reseachers found that the method performed favourably over traditional HMM optimization methodologies such as maximum likelihood and spectral clustering. However, Nystrup et al. (2020b) failed to investigate whether the jump model would be an improvement over Gaussian HMMs in terms of reproducing the stylized facts. As such, this thesis will, among other things, investigate how the well the jump model reproduces the stylized facts, particularly volatility clustering. Furthermore, Nystrup (2014) found that Markov-switching mixtures are able to reproduce a large degree of the aforementioned stylized facts, hence the utilization of HMMs can serve as the foundation for regime-based asset allocation strategies. 


\subsection*{Time-varying parameters}
The introduction of time-varying parameter models to capitalize on the time-varying risk premiums described
by Fama and French (1989) dates back to Quandt (1958), who pioneered by introducing an approach that would estimate a linear regression system with two regimes. Following his earlier work, Quant (1972) refined his techniques, and applied them to analyze disequilibria in the housing market. In subsequent research, Goldfeld and Quandt (1973) introduced Markov-switching regression, which coincided with similar research being conducted in the field of speech recognition successfully. The origination of these models can be back-traced to Baum and Petrie (1966) and Baum et al. (1970), who developed probability variables used in the HMM algorithms and thus laid the groundwork for the works of Dempster et al (1977) and Rabiner (1989), who uncovered the final algorithms that are used in HMMs today. As such, HMMs have widespread use in fields as signal-processing, however, the introduction of Markov-switching models to economics and finance was driven by research conducted by Hamilton (1989).

Regime shifts in the data-generating process leads to time-varying parameters, however, it is also evident that the parameters characterising each regime state as well as the transition probabilities also change over time. As such, it is not realistic to assume that the parameters of the conditional distribution only jump between a fixed number of constant values (Nystrup, 2017). In a study by Rydén et al. (1998) the researchers founds that the optimal number of states as well as the parameters have changed significantly across time. Therefore, based on these findings, to capture these dynamics with a stationary model results in highly complex models with a wide array of variables and parameters that need to be estimated. A preferred alternative involves utilizing recursive and adaptive techniques. These techniques have been present in engineering literature since the early 1980s, however, these estimation techniques have recently gained traction, specifically for HMMs (Khreich, 2012). As such, it Khreich (2012) showed by allowing the model parameters to be time-varying much more complicated dynamics can be captured by the models and this does not increase the overall number of parameters.

\subsection*{Dynamic asset allocation and portfolio optimization}
Asset allocation encompass the decision of how to allocate and distribute capital among a set of major asset classes and although the behavior of different assets classes are subject to variation across shifting economic regimes, no individual asset class dominates across all regimes. As such, rather than adopting to macroeconomic shifts, SAA investors seek to develop static "all-weather portfolios". However, if economic conditions can be modelled with high reliability and these findings are persistent, then dynamic asset allocation (DAA) should add value when compared to SAA strategies. The purpose of DAA is to exploit favourable economic conditions while mitigating draw-downs and losses in high volatility markets. As such, the objective of DAA is not to forecast and predict economic regimes but rather to identify the current regime and build investment strategies consequent on the identified regimes. As such, DAA is much more restrictive than SAA regarding the investment opportunity set, since illiquid assets such as real estate, private equity, infrastructure etc, are difficult to dynamically optimize (Nystrup, 2014).

Regime-based investing further separates itself from tactical asset allocation (TAA) since the latter has a shorter investment-horizon spanning from weeks to months at the longest, and the strategy is primarily driven by valuation considerations. Regime-based investing targets a longer time-horizon (e.g. months or years) and the strategies are driven by shifting economic fundamentals. As such, a regime-based investment provide the flexibility for fund managers to adapt to shifting economic conditions, hence regime-based investing straddles a middle ground between SAA and TAA (Sheikh \& Sun, 2012).

Dynamic asset allocation aims to benefit from volatility persistence, since risk-adjusted returns are considerably lower during turbulent periods, irrespective of the source of turbulence (Kritzman \& Li, 2010). The negative correlation between returns and volatility can be understood by investigating the investors' changing attitude towards risk. Since high volatility regimes often are associated with increasing risk-aversion, these highly volatile regime environments are likely to be accompanied by falling asset prices (Nystrup, 2017).

It follows from Zakamulin (2014) that the DAA framework can be split into a rule and optimization-based methodology. The rule-based approach can be amplified as defining a set of optimal portfolios, consisting of different capital allocation to a variety of asset classes, for each potential economic regime, and these portfolios are then traded based on whether the volatility is above or below a specified threshold. As such, RBAA is a subset of the rule-based methodology, since the asset allocation is dependent on the inferred economic regimes. Fornaciari \& Grillenzoni (2017) proposed a development to the theory by suggesting to estimate the model parameters and the portfolio weights recursively and simultaneously. However, even though one can optimize the portfolio weights in sample, doing so does not guarantee an optimal decision rule for the problem at hand. As a natural consequence a larger number of different specifications might have to be tried, before a decision rule with satisfactory performance can be uncovered. Yet, when testing a high degree of different specifications in-sample, one increases the risk of of overfitting and thus inferior out of sample performance. Further, it can be argued that a static decision rule is hardly optimal when the underlying model used for regime inference is time varying, as in Bulla et al. (2011) and Nystrup (2017).

Furthermore, although DAA is a multi-period optimization problem, it is often attempted approximated by a variety of single-period optimizations, thereby making it impossible to correctly adjust the solutions for the impact of trading costs, constraints and time-varying forecasts. Based on the litterature published by Mossin (1968) and Merton (1969) subsequent findings regarding multi-period portfolio optimization is based on dynamic programming, which makes it possible to establish a recursive framework that updates information as a sequence of trades is chosen (Gârleanu \& Pedersen, 2013). However, dynamic programming falls short in practise when it comes to trade selection due to the curse of dimensionality (Bellman, 1956). In later studies Boyd et al. (2014) aimed at circumventing the issues related to dynamic programming and thus solve the stochastic control problem underlying DAA by using model predictive control (MPC). The idea presented by Boyd (2014) involves optimizing the portfolio on a daily basis, based on financial return forecasts and with respect to the inferred regime probabilities, while taking transaction and holding costs as well as risk aversion into account. As such, it appears that MPC makes for a promising alternative to the traditional static all-weather decision rule which used to dominate the literature (Ang \& Bekaert, 2002). In addition, there are computational advantages to using MPC in setups where the financial return forecasts are updated whenever a new observation becomes available, since the optimal
control actions are reconsidered on a daily basis anyway (Nystrup, 2017). Furthermore, MPC allows for straight-forward implementation of different types of costs and constraints that are crucial for portfolio management. Therefore, the methodology build upon portfolio theory put forward by Markowitz (1952) highlighting that the objective is to uncover an optimal risk-return trade-off. Even though the mean-variance criterion is the most common approach in terms of portfolio selection (Kolm et al. 2014), the MPC methodolgy allows for alternative risk measures such as maximum drawdown and expected shortfall. As such, the thesis will focus on implementation of the MPC approach rather than simply switching between a predefined set of portfolios based on the regime sequence.  



\subsection*{Potential estimation biases}
1. Survivorship bias
2. Data quality
3. 

\textbf{The introduction is omitted from this version. Instead I have briefly given an overview of the sections in the following bullet points.}

\subsection*{Thesis statement}
The purpose of this thesis is to compare the performance of a regime-based asset allocation strategy under realistic assumptions to equally-weighted as well as static all weather portfolios. Since asset allocation has been shown to be the most important determinant of portfolio performance, it is clearly relevant whether a dynamic strategy can outperform a static strategy. The relevance is supported by the large amount of articles written on the subject in which particularly pensions funds and large institutional investors are looking to increasingly incorporate dynamic decision-making in terms of their portfolios, both related to return enhancement but also for risk management.

The asset classes considered in the analysis comprise a variety of financial asset including stocks, bonds, infrastructure and alternatives as well as commodities like gold and oil. A wide array of assets is taken into consideration since large institutional investors, like pension funds, are increasing their capital allocation towards alternatives and infrastructure assets, and the composition of assets is complex enough to provide diversification benefits. 

The analysis will include the following steps:

1. Review of stylized facts, accompanied by an analysis of temporal and distributional properties of S\&P500.

2. Introduction to data model - Hidden Markov Models (HMM).

3. Description of estimation methods used to train HMMs. We generally train the models using either maximum likelihood estimators or jump estimators.

4. Simulation study on model convergence. The purpose of this section is analyze the convergence of the various estimators used.

5. Reproducing stylized facts. In this section we train the estimators on real data to analyze how well they can reproduce stylized facts.

6. Dynamic portfolio optimization using model predictive control (MPC). In this section we explain the method behind our proposed dynamic asset allocation strategy.

7. Results.


%Python is used with regards to modeling and forecasting. The setup of the code and the associated file-management is controlled through an online Github and the PyCharm project-management software. For optimization, Python and the library CVXPY (Diamond and Boyd 2016) is used.
