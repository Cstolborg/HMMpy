\section{Introduction}
\label{section: introduction}

\textbf{Størstedelen her skal skrives om så det matcher mere med opgaven. Generelt er det vigtigt at tingende bliver bundet op på det vi rent faktisk undersøger. Det er fint at lave et (kort) litterature review indenfor hver underoverskrift, men det SKAL bindes op på hvordan det benyttes i opgaven ellers er det værdiløst.}

\textbf{Vi skal også lige finde ud af hvor langt det her egentligt skal være. Tænker noget ala:}

\begin{itemize}
    \item Generelt intro til DAA, regimed-based asset allocation og brugen af HMMs til dette - ca. 1 side.
    \item Thesis goals. I dette afsnit beskriver vi målet ved thesen. Den generelle opbygning gennemgås og der gives delkonklusioner fra hver subsektion - ca 1-1.5 side.
    \item Evt. yderligere beskrivelser af HMMs litterature reivew, valg af time frequency etc etc.
\end{itemize}

\begin{enumerate}
    \item Motiver emnet. Kort gennemgang af nogle af de studier vi lægger os tæt op af, hvilket interessante findings de havde og hvordan man kan bygge videre på dem - dvs. mangler - Bulla, Ryden, Nystrup - 1 side.
    \item Thesis statement. Fokus på sektion 5-7. Beskriv først hvorfor hvert af de 3 emner er interessante og hvordan de spiller sammen (1 paragraf) og gennemgå dem derefter en ad gangen (1 paragraf per emne).
    \item Litterature review
\end{enumerate}

The behavior of financial markets has been, and continues to be, subject to abrupt changes. As such, the GFC of 2008 showed that means, volatility and correlation patterns in stock returns changed considerable in the years to follow and the investors' risk appetite decreased across asset classes (Uhlenbrock, 2009). Furthermore, the GFC demonstrated that diversification is insufficient to mitigate large drawdowns in periods where it is needed the most since the correlation between risky assets tend to strengthen during periods of high market volatility (Pedersen 2009). 

Despite this, asset allocation which is defined as the choice of weights between major asset classes, has proven to be the most important determinant regarding portfolio performance (Brinson et al. 1986, Ibbotson \& Kappland 2000). As such, time-varying investment opportunities suggest that the weights underlying the portfolio allocation should be adjusted as new information arises (Bansal et. 2004). One of the methodologies that most recently has been proposed is to identify a number of economic regimes, which then determine the asset allocation procedure. 

A universal and macroeconomic way to identify regime changes is through Composite Leading Indicators (CLI), which provide signals of turning points in the business cycle by capturing fluctuations of the economic activity around its long term level (Hesselholt \& Knuthsen, 2009). Yet, these indicators tend to lag behind the economic activity due to the delay in how economic data is released. Therefore, there is a high risk that the regime change has already happened, when the data becomes available for analysis (Nystrup, 2014). As such, regime-switching models, based on daily financial returns series, become a reliable alternative as changes to regimes can be identified almost instantly due to the fact that new economic information is being priced into financial assets continuously. This serves as one of the key arguments to why this thesis will concentrate on the usage of HMMs rather than multivariate logistic economic regression analysis. As such, the following section will provide a review of the developments within regime shifts in regards to portfolio theory.

\subsection{Thesis statement}
Based on the introduction above, the motivation and overall purpose of this thesis is rooted in an objective to compare the performance of a regime-based asset allocation strategy to that of equally-weighted as well as static all-weather portfolios. However, before one can compare the results through a portfolio exercise, some additional analysis are needed. As such, the thesis will originate by introducing the financial data that will be used in terms of model estimation, while also providing an overview of the key temporal and distributional properties related to the data. Secondly, the thesis will introduce the general notation and assumptions regarding HMMs which will be followed by en estimation section highlighting the mathematics of two distinct approaches known as maximum-likelihood estimation as well as jump estimation. 

Having provided an overview of the estimation procedures the thesis will proceed by conducting a simulation study designed to test how the different parameters of the models converge to the true values when simulated from correctly and incorrectly specified distributions as well as from a variety of sample lengths. This is done in order to validate or disqualify the models and determine whether there is a substantial difference between the estimation procedures in terms of how likely the models are to be subject to large estimation errors. Furthermore, the section will provide an in-depth analysis of parameter tuning, particularly in regards to the jump model. Once a sufficient simulation study has been conducted and provided that the models obtain sufficient results the thesis will analyse whether the models are able to reproduce the stylized facts introduced by Granger \& Ding (1995b) in which a direct comparison will be drawn to the studies conducted by Rydén et al. (1998) and Bulla (2011). This is one of the crucial parts of the thesis because if the models fail to reproduce the stylized facts well, it becomes needless to perform a portfolio exercise rooted in the estimates of the models. However, assuming that the the estimated models are able to sufficiently reproduce the stylized facts, the thesis will conclude by conducting a portfolio exercise in which the overall objective of the thesis is uncovered, namely whether RBAA combined with MPC results in supperior risk-adjusted returns compared to an equally weighted portfolio as well as a static all-weather portfolio strategy, thereby answering the question if profitable investment strategies exist. As such, the analysis will be divided into the following steps.


%Python is used with regards to modeling and forecasting. The setup of the code and the associated file-management is controlled through an online Github and the PyCharm project-management software. For optimization, Python and the library CVXPY (Diamond and Boyd 2016) is used.

\subsection{Regime shifts}
Regime shifts are prevalent across financial markets as well as in the behavior of macroeconomic variables. Further, regime shifts are typically either recurring, illustrated through recessions and expansions, or permanent in which they take the form as structural breaks in the economy (Ang \& Timmermann, 2012). In relation to this, Campell (1999) and Cochrane (2005) observed regimes in financial markets are related to the general economic business cycle. As such, the HMMs utilized in this thesis will rely on financial market data as opposed to macroeconomic variables. This is due to the fact that when operating under the assumption that financial markets are efficient, the outlook for the future economic activity should be captured in the development of asset prices (Siegel, 1991). Given this assumptions it is striking that most studies regarding DAA and RBAA consider monthly rather than daily return series. 

The first to consider the impact of regime changes on asset allocation was Ang \& Bekaert (2002). As such, they found that RBAA has the potential to reduce drawdown and increase profit over simple rule-based rebalancing. Several studies have followed the original work of Ang \& Bekaert (2002), including Ang \& Bekaert (2004), Bauer et al.
(2004), Guidolin and Timmermann (2007), Bulla (2011) and Kritzman et al. (2012). The majority of the listed research considered DAA across assets including stocks, bonds as well as a risk-free asset, however, the proposed solutions often lead to allocation changes that exceeds the investment mandate of the portfolio manager. 

It is important to underline and stress that regime shifts are an abstraction of reality that aims to capture and provide a tangible association to changes in economic variables. As such, models that are specified with a fixed number of recurring regimes, which will be discussed further below, are just one way to model and capture these economic changes (Ross et al. 2011). As such, previous research have concluded that when coupling financial data with regime-switching models like HMMs one can establish a link to the general business cycle, hence the methodology can be used for asset allocation purposes. However, a crucial aspect when dealing with financial models involves their ability to reproduce a set of stylized facts introduced by Granger \& Ding (1995b) and further elaborated by Cont (2001). Since this will be an instrumental part of the thesis, the following section will provide a literature review as well as initial thoughts in terms of the stylized facts of financial returns and HMMs.


\subsection{Literature review of HMMs and stylized facts }
The introduction of time-varying parameter models to capitalize on the time-varying risk premiums described by Fama and French (1989) dates back to Quandt (1958), who pioneered by introducing an approach that would estimate a linear regression system with two regimes. Following his earlier work, Quant (1972) refined his techniques, and applied them to analyze disequilibria in the housing market. In subsequent research, Goldfeld and Quandt (1973) introduced Markov-switching regression, which coincided with similar research being conducted in the field of speech recognition simultaneously. The origination of these models can be back-traced to Baum and Petrie (1966) and Baum et al. (1970), who developed probability variables used in the HMM algorithms and thus laid the groundwork for the works of Dempster et al (1977) and Rabiner (1989), who uncovered the algorithms underlying the maximum likelihood estimation procedure of HMMs. As such, HMMs have widespread use in fields as signal-processing, however, the introduction of Markov-switching models to economics and finance was driven by research conducted by Hamilton (1989)

It is well known within quantitative finance that a Gaussian distribution provides a poor fit to most financial returns. This is evident as the Gaussian distribution fails to capture the fat tails exhibited by returns as well as the volatility clustering. However, mixture distributions provide a better fit since they are able to reproduce the leptokurtosis and skewness prevalent in financial return series (Cont, 2001). In order to capture both the distributional and temporal properties of financial returns, an extension to the traditional Markov switching mixture, known as HMMs have been particular promising (Bulla, 2011). The model was first introduced to the area of financial economics by Hamilton (1989) and it has since gained stronghold in quantitative finance. Furthermore, one of the key aspects with HMMs as that the inferred states can directly be linked to the different phases in the economic cycle (Ang \& Timmermann, 2012). Therefore, the fact that the model allows for interpretability of the states combined with its ability to reproduce the stylized facts makes it highly popular. 

After the introduction by Hamilton in 1989 it was shown by Rydén et al. (1998) that HMMs comprise the ability to reproduce most of the stylized facts put forward by Granger \& Ding (1995b). However, HMMs fall short when attempting to replicate the slowly decaying autocorrelation function of absolute daily returns, which is also referred to as volatility clustering (Cont, 2001). In an attempt to improve the fit of HMMs as well as their ability to reproduce the long memory property of financial returns, a variety of papers have followed and extended the findings related to Gaussian HMMs by Rydén et al. (1998). As such, Bulla \& Bulla (2006) found that the HMMs insufficient ability to reproduce the long memory property can be explained by the implicit assumption of geometrically distributed sojourn times in a particular hidden state. To circumvent this limitation, Bulla \& Bulla (2006) considered hidden semi Markov models (HSMM), which make it possible to explicitly model the sojourn time distribution for each state. Upon completion of their research, Bulla \& Bulla (2006) found that HSMMs are better at reproducing the long term memory of daily financial returns, however, they failed to specify an optimal sojourn time distribution. Bulla (2011) further expanded on the earlier research by showing that HMMs with t-distributed components reproduce a broad array of the stylized facts as well or better than Gaussian HMMs. 

The theory of regime-switching models and HMMs has most recently been expanded by, Nystrup et al.(2020b) who combined the jump framework of Bemporad et al. (2019) with the temporal features used by Zheng et al. (2019) in order to train HMMs. The inclusion of a jump-penalizer provides influence over the transition rates in the predicted state sequence and the reseachers found that the method performed favourably over traditional HMM optimization methodologies such as maximum likelihood and spectral clustering. However, Nystrup et al. (2020b) neglected to investigate whether the jump model would be an improvement over Gaussian HMMs in terms of reproducing the stylized facts. As such, this thesis will, among other things, investigate how the well the jump model reproduces the stylized facts and whether or not it outmatches the clasical maximum likelihood estimation approach of HMMs. 

\subsection{Dynamic asset allocation and portfolio optimization}
Asset allocation encompass the decision of how to allocate and distribute capital among a set of major asset classes and although the behavior of different asset classes are subject to variation across shifting economic regimes, no individual asset class dominates across all regimes. As such, rather than adopting to macroeconomic shifts, SAA investors seek to develop static "all-weather portfolios". However, if economic conditions can be modelled with high reliability and these findings are persistent, then DAA and RBAA should add value when compared to SAA strategies. As such, the purpose of DAA and RBAA is to exploit favourable economic conditions while mitigating draw-downs and losses in high volatility markets. This means that the objective of DAA and RBAA is not to forecast and predict economic regimes but rather to identify the current regime and build investment strategies around this. As such, DAA and RBAA are much more restrictive than SAA regarding the investment opportunity set, since illiquid assets such as real estate, private equity and infrastructure are difficult to dynamically optimize (Nystrup, 2014).

It follows from Zakamulin (2014) that the DAA and RBAA framework can be split into a rule and optimization-based methodology. The rule-based approach can be described as defining a set of optimal portfolios, for each potential economic regime, and these portfolios are then traded based on the estimated regime from a HMM. However, even though one can optimize the portfolio weights in-sample, doing so does not guarantee an optimal decision rule for the problem at hand Fornaciari \& Grillenzoni (2017). As a natural consequence a larger number of different specifications might have to be tried, before a decision rule with satisfactory performance can be uncovered. Yet, when testing a high degree of different specifications in-sample one increases the risk of overfitting and thus inferior out-of-sample performance. Further, it can be argued that a static decision rule is hardly optimal when the underlying HMM used for regime inference is time varying, as in Bulla et al. (2011).

In later study by Nystrup (2017) the use of model predictive control (MPC) was introduced to circumvent the issues in terms of rebalancing to a static set of predefined in-sample weights. The idea, originally presented by Boyd (2014), involves optimizing the portfolio on a daily basis, based on financial return forecasts and the inferred economic regimes, while taking transaction and holding costs as well as risk aversion into account. As such, it appears that MPC makes for a promising alternative to the traditional static all-weather decision rule which used to dominate the literature. As such, this serves as one of the key reasons as to why this thesis relies on the MPC framework in the portfolio exercise, because when the HMMs are updated on a daily basis it makes perfect sense to check whether the optimal portfolio allocation has changed as well and whether it is economically feasible to update the allocation if such a change has occurred. As such, the HMMs and the MPC framework is interconnected since they rely on the same data frequency in their estimation procedure. Lastly, the MPC framework will be utilized due to the fact that it allows for straight-forward implementation of different types of costs and constraints that are crucial for portfolio management. Therefore, the methodology build upon the portfolio theory put forward by Markowitz (1952) highlighting that the objective is to uncover an optimal risk-return trade-off. Even though the mean-variance criterion is the most common approach in terms of portfolio selection (Kolm et al. 2014), the MPC methodology allows for alternative risk measures such as maximum drawdown and expected shortfall. As such, the thesis will focus on implementation of the MPC approach rather than simply switching between a predefined set of portfolios based on the regime sequence.  


\subsection*{Time horizon and frequency of data}
\label{subsection: Data frequency}

When utilising financial market data to uncover economic regime changes, a natural questions arises in terms of which data frequency to use. As described, the objective of DAA is to rebalance the portfolios once a regime shifts has occurred, hence if the regime detection relies on too infrequent data, there is a high probability that several regime shifts will remain hidden. As such, data frequencies longer than a month is not considered. Furthermore, as previously mentioned, broad macroeconomic data is widely available, however, the data is typically characterised by a data frequency of months or even quarters. Therefore, the data frequency of macroeconomic variables propose a challenge, as historic events has shown that economic regime changes can happen swiftly, evident by the recent COVID-19 recession. As such, monthly data compared to e.g. daily data, greatly increases the risk of slow and insufficient detection of economic regime shifts.
 
Despite of the reasoning just outlined, it should be noted that the use of daily data presents some challenges as well. This is due to the fact that daily returns contain a lot of noise and extreme observations, which are evened out on a monthly basis. Consequently, long-horizon returns tend to be more closely approximated by a Gaussian distribution compared to returns for shorter time horizons (Campbell et al. 1997). As such, short-term data frequencies complicates the modeling significantly, which can lead to sporadic predictions and less persistence in the uncovered economic states. In addition, the use of daily return frequencies makes the aforementioned link to macroeconomic data more difficult to justify, and therefore it becomes challenging, from a macroeconomic perspective, to argue that economic regime shifts occur since there is no tangible macroeconomic evidence to support this before the data gets collected and released. Yet, despite the complications associated with short-term data frequencies, the arguments for considering daily as opposed to monthly data frequencies are compelling. 

Furthermore, despite the aforementioned issues with sporadic predictions due to daily data frequencies, Bulla et al. (2011) argued that by relying on daily data it becomes feasible to apply filters that increase confidence whenever a regime change has been detected. As such, research cements the possible of implementing a waiting scheme, in which the portfolio manager would wait several days, when a regime shifts has been detected, before changing the portfolio allocation, thereby minimizing the risk of re-allocating capital based on a wrong signal. In addition, the use of model predictive control (MPC) in section \ref{Subsection: Model predictive control} makes daily data frequencies more justified, since it is not possible to make better return predictions for the assets than their long-term average. As such, looking only a limited number of days into the future is not just an approximation necessary to make the optimization problem computationally feasible, it is also reasonable due to the nature of financial returns.

Bulla (2011) further argued that the use of daily data frequencies increases the amount of data available for markets characterised by a short lifetime, however, many financial indices are old and for these older indices monthly data is more available than daily data. Unsupervised machine learning models, such as HMMs, require large amounts of data to be trained in order to achieve a desired and comfortable level of predictability and persistence, hence the usage of these heavy data-driven models serves as an argument for relying on monthly data as opposed to daily data. Yet, this thesis favors early detection, due to the reasoning outlined above. As a result, daily data, will be used in the analysis. Additionally, the optimal time horizon used for training the model is debatable, however, it should at least span the time required for a financial cycle to unfold in order to include the performance of DAA strategies in several economic environments. Finally, it should be acknowledged that, the longer the data horizon the more questionable it becomes whether stationarity of the data-generating process can be assumed (Bulla et al. 2011). The following sections will include details on the S\&P 500 index.