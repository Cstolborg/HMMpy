\newpage
\section{Index data and stylized facts of financial returns}
\label{section: Data}

\textbf{Jeg syntes det med time frequency skal rykkes til intro, returns skal rykkes til sektion 6 og dist+temp properties skal i appendiks.}

In this chapter, the data used to uncover the economic regimes are introduced. The primary index that will be used to estimate the parameters of the HMM is the S\&P 500 as this is one of the most renowned and liquid stock indices containing some of the largest blue chip companies in the US, and its data dates back long enough to be sufficient for training unsupervised machine learning models. Initially, the choices and associated consequences related to time horizon and optimal observation frequencies are discussed. Secondly, once the data of the S\&P 500 index has been introduced and reviewed its distributional and temporal properties are explored in section \ref{subsection: distributional properties} and \ref{subsection: temporal properties}. The objective of the chapter is thus to provide an overview of the data that serves as potential usage for regime detection along with its statistical properties. The reader should note that an extensive analysis of the stylized facts associated with the empirical data as well as the estimated models will be conducted in section \ref{Section: Stylized facts}.  



