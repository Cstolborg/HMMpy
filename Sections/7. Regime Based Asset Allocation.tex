\newpage

\section{Regime Based Asset Allocation}
Optimal portfolio allocation boils down to essentially maximizing the returns generated for a given amount of risk. The portfolio manager who successfully and consistently does so, will have more money to manage and more satisfied investors. Generally speaking there are three approaches when it comes to portfolio selection that sophisticated institutional investors utilize (Nystrup, 2017). The first approach consists of diversification, which entails increasing risk-adjusted returns by forming optimal portfolios on the basis of imperfectly correlated assets. This is also known as static asset allocation, since the portfolio is constructed to capitalize on bullish market periods while providing protection in bearish market periods. As such, static asset allocation is considered an all-weather portfolio (Markowitz, 1952). However, the challenge with static all-weather portfolios is that in order for them to remain efficient they must be continuously re-balanced and this trading is subject to trading costs, thereby negatively impacting returns. Furthermore, and perhaps more importantly, the financial crisis of 2008 demonstrated that diversification is not sufficient to avoid large drawdowns (Nystrup, 2017). As such, diversification fails when investors need it the most since the correlations between risky assets have a tendency to strengthen when markets are characterised by high volatility (Pedersen, 2009). Furthermore, the large drawdowns in periods of high market volatility challenge investors' psychological and financial tolerance which will ultimately lead to redemption of funds and firing of portfolio managers. As such, a reasonably low maximum drawdown (MDD) is critical for the success of any portfolio.

In order to counter the limitations of static asset allocation, particularly in markets with high volatility, portfolio managers and institutional investors began to switch from static all-weather portfolios to dynamically optimizing and rebalancing the weights of the traded assets. This is known as dynamic asset allocation (DAA). As such, SAA involves setting an asset mix for the long-term with periodic adjustments, for instance yearly, while DAA involves frequently adjusting the portfolio allocation to respond to changes in the market conditions. Following this, the predominant approach in previous studies has been to specify a static decision rule for changing the allocation based on the state of the economy. This approach was implemented in a study by Nystrup (2014) who showed that combining dynamic rebalancing with the ability to uncover economic regimes through HMMs, in order to adjust portfolio weights as new information arrives, thereby allowing investors to take advantage of favorable market conditions while withstanding and limiting the downside in high volatility market periods, result in higher risk-adjusted returns compared to traditional SAA and DAA strategies. This method is referred to as regime-based asset allocation (RBAA) and several studies have confirmed that RBAA provides higher risk-adjusted returns when bench-marked against rebalancing to static weights while dramatically reducing drawdowns in periods of high market volatility. For additional information on RBAA and its development the authors refer to Ang \& Bekaert (2004), Guidolin \& Timmermann (2007), Bulla (2011) and Nystrup et al. (2015a, 2017a). 

One of the big issues related to the methodology presented by Nystrup (2014) is that the conditional portfolio weights for a specific economic regime are optimized in sample, however, there is no guarantee that the portfolio weights are optimal when the regime change actually occurs. The natural disadvantage is that a large number of different portfolio weight specifications might have to be tried, in order to uncover a decision rule with good performance. However, testing many different specifications increases the risk of inferior performance out of sample due to overfitting and it can also be argued that a static decision rule is hardly optimal when the underlying HMM used for regime inference is time varying (Nystrup, 2017).

An alternative approach to simply switching between a static decision rule is to dynamically optimize the portfolio while adjusting for transactions costs, risk aversion as well as a variety of other constraints. This methodology is known as model predictive control (MPC). One of the great strengths of the MPC framework is its capability to solve control problems with several constraining factors in a computationally feasible manner. One of the key points of analysis in regards to the MPC framework going forward will be the transactions costs. This is because they are an instrumental consideration for portfolio managers and investors when comparing the performance of SAA against RBAA since frequent rebalancing can offset the potential excess return of a dynamic strategy. 

Furthermore, as also mentioned, the thesis operates through a framework in which asset returns are modeled by a two-state Gaussian HMM with time-varying parameters. This is due to the fact that HMMs are a more realistic description of asset price dynamics than a linear factor model with constant variance. In addition, it was shown in section \ref{Section: Stylized facts} that HMMs are well suited to capture the stylized behavior of financial returns, including volatility clustering, leptokurtosis and time-varying correlations (Rydén et al. 1998). Lastly, HMMs are well suited to describe the abrupt changes in financial market that arise due to changes in the state of the economy.   

The task of deriving optimal portfolios in a RBAA setting is a multi-period problem, however, it is often approximated by a sequence of single-period optimizations, thereby making it impossible to properly account for the impact of trading constraints and time-varying forecasts (Nystrup, 2017). Following the research of Gârleanu \& Pedersen (2013), multi-period portfolio selection has predominantly been based on dynamic programming. However, actually carrying out dynamic programming for trade selection is impractical, except for some cases with limited assets due to the “curse of dimensionality” (Boyd et al. 2014). As a natural consequence of this, most previous studies only include a small number of assets while keeping the constraints extremely simple. This thesis mitigates this limitation by expanding on the MPC framework introduced by Boyd et al. (2017) thereby making it possible to consider a large numbers of assets while imposing additional constraints related to trading costs, maximum drawdown and position sizing. The following sections will introduce the MPC framework as well as the underlying constraints.

\subsection{Model Predictive control}
\label{Subsection: Model predictive control}

The overarching idea behind the MPC approach is to dynamically optimize the portfolio based on forecasts of means and variances of returns at discrete time horizons. By relying on a 2-state Gaussian HMM means that the forecasts are mean-reverting and only change when the regime probabilities change (Nystrup, 2017). As such, the forecasted means and variances serve as inputs to the multi-period portfolio optimization problem. Intuitively, this means that every day a decision has to be made whether or not to change the current portfolio allocation, knowing that the decision will be reconsidered the next day with new inputs. The possible benefits from changing the asset allocation in the portfolio should be traded off against risks and costs. 

Naturally, there is a limit to how far into the future it remains meaningful to make predictions about the means and variances of returns. As such, for long horizons it is not possible to make better predictions than the long-term mean and variance, hence the forecasted mean and variance converge to their stationary values when the forecast horizon becomes large enough. This means that relying on a small time horizon in terms of the predictions of means and variances of returns is not just an approximation necessary to make the optimization problem computationally feasible, it also seemingly reasonable (Nystrup, 2017).

\subsubsection{Stochastic control formulation}
The fact that the multi-period asset allocation optimization problem is formulated as a stochastic control problem is based on the work conducted by Boyd et al. (2017). Define $w_t$ $\in$ $\mathbb{R}^{n+1}$ as the portfolio weights at time \textit{t} in which $(w_t)_{i}$ is the fraction of the total portfolio $V_t$ invested in asset \textit{i} at time $t$. This definition means that $(w_t)_{i} < 0$ is a short position in asset \textit{i}. The weight $(w_t)_{{n+1}}$ represents the fraction held in risk-free asset. By definition, the weights of all the assets comprising the portfolio sum to 1, i.e. ${1}^Tw_t = 1$, in which $\mathbf{1}$ is a column vector with 1 in all entries. 

A natural objective for investors is to maximize the present value of the future expected returns over a given investment horizon $T$, on a risk-adjusted basis, in which transactions and holding costs are taken into consideration. As such, the following objective can be defined.

\begin{equation}
\begin{split}
    E\left[\sum_{t=0}^{T-1}\eta_{t+1}\left(r_{t+1}^Tw_{t+1} -\gamma_{t+1}\psi_{t+1}(w_{t+1})\right) \right. \\
    \left. -\eta_t\left(\phi_t^{trade}(w_{t+1}-w_t)+\phi_t^{hold}(w_{t+1})\right)
    \vphantom{\sum_{t=0}^{T-1}} \right]
    \label{eq: MPC_stochastic}
\end{split}
\end{equation}

The objective states that the expectation is taken over the sequence of returns $r_1, r_2...., r_T \in \mathbb{R}^{n+1}$ conditioned on all past observations. Further, $\psi_t:\mathbb{R}^{n+1} \rightarrow \mathbb{R}$ is a risk function, $\gamma_t$ is a risk-aversion parameter which is used to adjust and control the relative importance of return versus risk, $\phi_t^{trade}: \mathbb{R}^{n+1}\rightarrow\mathbb{R}$ serves as a cost function for trading, $\phi_t^{hold}: \mathbb{R}^{n+1}\rightarrow\mathbb{R}$ serves as a holding cost function and $\eta \in (0,1)$ is the discount factor, usually equal to the inverse of $1+r_f$ where $r_f$ is the risk-free rate.

\textbf{Describe the bounds of $\eta$. Can discount factor not exceed 1 if rf<=0?}

\subsubsection{Deterministic control formulation}

In order to determine which trades to make the MPC framework replaces all future unknown quantities by their forecasted values over a forecast horizon $H$. As such, the future returns, which by nature are unknown, are replaced by their forecasted mean values $\hat{\mu}_{\tau|t},\tau = t+1,....,t+H$, in which $\hat{\mu}_{\tau|t}$ is the forecasted mean value of returns for the time period $\tau$ conducted at time $t$. This property turns the stochastic optimization problem from equation \ref{eq: MPC_stochastic} into the deterministic optimization problem 
\begin{equation}
\begin{split}
    \max_{w_{t+1},\ldots,w_{t+H}} \sum_{\tau=t+1}^{t+H}(\hat{\mu}_{\tau|t}^Tw_\tau-\hat{\phi}_{\tau|t}^{trade}(w_\tau-w_{\tau-1})-\hat{\phi}_{\tau|t}^{hold}(w_\tau)-\gamma_\tau\hat{\psi}_{\tau|t}(w_\tau))
    \label{eq: maximizing objective MPC}
    \\
    \text{Subject to } \mathbf{1}^Tw_\tau=1, \quad \tau = t+1,...,t+H,
\end{split}    
\end{equation}
where $\hat{\phi}^{trade}$ and $\hat{\phi}^{hold}$ are trading and holding cost functions, which will be further defined in subsequent sections. Note that $w_t$ is not a variable in the control formulation since it is a current known quantity. 

Therefore, when solving the optimization problem from equation \ref{eq: maximizing objective MPC} the result is an optimal sequence of weights $w_{t+1}^*,..., w_{t+H}^*$. As such, the difference between the weights at each time step $t$ represents the future trades over the planning horizon $H$, however, this only holds upon fulfillment of the unrealistic assumption that all future unknown quantities will match their forecasted values. In order to minimize the risk of making inappropriate capital allocations, only the first predicted trade $w_{t+1}^*-w_t$ will be executed. At the next time step $t+1$ the process will be repeated. The reader should note that the planning horizon $H$ can be much shorter than the investment horizon $T$ and this is indeed why the discounting factor is ignored in equation \ref{eq: maximizing objective MPC} as opposed to equation \ref{eq: MPC_stochastic} 
(Nystrup, 2017). The procedure of the MPC approach for multi-period portfolio selection is depicted in the algorithm below. 

\begin{algorithm}[H]
\BlankLine
1. Update model parameters based on the most recent observation
\Indm
\BlankLine
2. Forecast future values of all unknown quantities $H$ steps into the future
\BlankLine
3. Compute the optimal sequence of weights $w_{t+1}^*,...,w_{t+H}^*$ based on the current portfolio $w_t$
\BlankLine
4. Execute the first trade as $w_{t+1}^* - w_t$ and return to step 1 of the algorithm.\;
\BlankLine
\caption{MPC approach to multi-period portfolio selection}
\label{algo:MPC}
\end{algorithm}

As such, algorithm \ref{algo:MPC} provides a summary of the steps involved in the MPC solution to the multi-period portfolio selection problem. Furthermore, the MPC approach holds computational advantages when coupled with an HMM, since the optimal control action is considered anyway for whether or not to change the portfolio allocation i.e. depending on whether a state transition has occurred. This means that the portfolio manager might as well go ahead an update the derivation of the optimal portfolio allocation provided that it is economically viable to do so. The formulation of equation \ref{eq: maximizing objective MPC} is a convex optimization problem under the assumption that the risk function as well as the transaction and holding costs functions can be characterised as convex (Boyd \& Vandenberghe, 2004). The formulation of the multi-period portfolio problem as a convex optimization problem has some computational and intuitive advantages, since computational remedies as CVXPY can be used in a Python setting, thereby speeding up processing time significantly.

In the next sections, the use of HMMs in the forecasting procedure of algorithm \ref{algo:MPC} is explained. This is followed by a deep-dive of the different choices when defining the aforementioned risk and control functions in equation \ref{eq: maximizing objective MPC}. More specifically this involves the choices for how to set the risk aversion parameter $\gamma$, the risk function $\psi$ and the cost functions $\phi^{trade}$ \& $\phi^{hold}$. Additionally, possible constraints on $w_t$ are discussed. Finally hyper-parameter tuning methods are reviewed.

\subsection{Forecasting with HMMs}
\label{section: forecasting MPC HMM}

Forecasting will be performed using the HMM estimators described throughout the previous chapters. As such, an identical rolling window length of \textbf{1700 observations} is utilized. In a Markowitz portfolio optimization, one relies on quantities including the assets' expected returns, variance and their covariance. Due to this, the procedure described here will structured as follows. Firstly, an HMM is estimated at time step $t$. Secondly, the state sequence over the rolling window is decoded using the Viterbi algorithm. This allows one to estimate each asset's sample mean return, standard deviation and covariance in each state. Based on those estimates, and by assuming that returns follow a multivariate log-normal distribution, the estimated parameters are transformed into normally distributed variables. Having computed conditional estimates of asset returns as a multivariate normal distribution, the unconditional distribution is found, for each time $t$ and each time $t+H$, by weighting the conditional distribution by the state probabilities $P(s_{T+H}=i | O)$ for all $i\in\K$. The procedure will be described in greater detail in the next paragraphs.

Having decoded the state sequence over the rolling window, the conditional means and covariances of returns are calculated from the estimated log returns. It should be noted that the log returns, for each state, are assumed to be i.i.d. and to follow a multivariate Gaussian distribution,
\begin{equation}
    \log(\mathbf{1}+ r_t) \sim N\left(\mu_{s_t}^{log}, \Sigma_{s_t}^{log}  \right)
\end{equation}
where $\mu_{s_t}^{log}$ and $\Sigma_{s_t}^{log}$ denote the conditional mean and covariance of the observed log-returns over the rolling window. $s_t$ denotes that the distribution of $\log(\mathbf{1}+ r_t)$, at time $t$, is dependent on the active state. Thus, using the formulas described by Munk (2019), the $i$th value in the vector $\mu_s$ and the $ij$th value in the matrix $\Sigma_s$ is calculated as
\begin{align}
    (\mu_{s})_i &= \exp \left\{(\mu_{s}^{log})_i + \frac{1}{2}(\Sigma_{s}^{log})_{ii} \right\} - 1,
    \quad i = 1,\ldots, n+1
    \\
    (\Sigma_{s})_{ij} &= \exp \left\{(\mu_{s}^{log})_i + (\mu_{s}^{log})_j + \frac{1}{2}\left\{(\Sigma_{s}^{log})_{ii} + (\Sigma_{s}^{log})_{jj} \right\} \right\} \\
    &\cdot \left\{ \exp \left\{ (\Sigma_{s}^{log})_{ij} \right\} - 1 \right\} \nonumber
    , \quad i,j = 1,\ldots,n+1
\end{align}
where $(\mu_{s})_i$ denote to the conditional mean of asset $i$ in state $s$, and $(\Sigma_{s})_{ij}$ denote the conditional covariance between asset $i$ and $j$ in state $s$. Having computed the first two conditional moments for all $n+1$ assets, it is possible to conduct forecasting. Since HMMs are probabilistic models, forecasting entails computing the forecast distribution at each time step $T+H$ (Zucchini, 2009).
\begin{equation}
    \hat\gamma(T+H) = P(s_{T+H}| O) = \hat\gamma(T) \cdot Q^H
\end{equation}
This means that once the model parameters are estimated at time $t$, the forecast distribution $H$ steps ahead is computed by multiplying the posterior distribution with the transition probability matrix $H$ times, thus producing the discrete distribution $\hat\gamma(T+H)$. Given the forecast distribution at each future time step, $\hat\gamma(T+H)$, and the conditional means and covariances, $(\mu_{s})_i$ and $(\Sigma_{s})_{ij}$, the unconditional distribution is
\begin{align}
    \mu_{T+H} &= \sum_s \hat\gamma_s(T+H) \mu_s,
    \quad s \in\K \label{eq:MPC_mean_uncond} \\
    \Sigma_{T+H} &= \sum_s \hat\gamma_s(T+H) \Sigma_s + \sum_s \hat\gamma_s(T+H) (\mu_s-\mu_{T+H})(\mu_s-\mu_{T+H})^T ,
    \quad s\in\K  \label{eq:MPC_var_uncond}
\end{align}
where $\mu_{T+H}$ and $\Sigma_{T+H}$ are the forecasted unconditional mean and covariance at time step $T+H$. Further, $\hat\gamma_s(T+H)$ denote the forecasted probability of being in state $s$ and $\mu_s$ and $\Sigma_s$ refer to the estimated conditional distribution in state s (which is not time-varying with the forecast). It should be noted that since the conditional distributions are distinct Gaussian distributions, then the resulting unconditional distribution described in equation \ref{eq:MPC_mean_uncond} and \ref{eq:MPC_var_uncond} is a mixture with non-Gaussian distribution (Frühwirth-Schnatter, 2006).


\textbf{Insert comments on where the forecast distribution goes to (stationary distribution) and other considerations.}

\subsubsection{Shrinkage of variance-covariance matrix}

As shown by Fiecas et al. (2017) and Nystrup et al. (2017) estimates of the covariance matrix quickly become unstable as one increase the number of assets. As such, when considering the trading strategy to be used in this analysis, in which trading potentially occurs each day, such a strategy can potentially not only cause large estimation errors and thus downright bad portfolio allocations but the framework can also recommend large frequent switches yielding unnecessarily inflated trading costs. The problem is even bigger in a regime-switching model, such as an HMM, because the data is further divided into states where some states may only occur very infrequently thereby causing sample sizes to drop. To alleviate the problem one can apply a Stein-type shrinkage estimator as proposed by Fiecas et al. (2017)
\begin{equation}
    \hat\Sigma_s^{shrink} = (1-\upsilon_s)\hat\Sigma_s + \upsilon_s tr(\hat\Sigma_s) \frac{1}{n+1} I_{n+1},
    \quad s \in\K
\end{equation}
where $\upsilon_s\in[0,1]$ is the shrinkage factor in state s, $tr(\cdot)$ is the trace of a square matrix, and $I_{n+1}$ is the $(n+1)\times (n+1)$ identity matrix.

In conclusion, the approach to forecasting means and covariance, as shown in step 2 of algorithm \ref{algo:MPC}), has been described. This will serve as input to the objective function in equation \ref{eq: maximizing objective MPC}. In the following sections some choices for cost functions are considered.

\subsection{Cost functions}

\subsubsection{Risk-averse-control}
Since the origination of quantitative portfolio research, through the introduction of mean-variance theory by Markowitz (1952), the most used risk-adjustment charge is depicted as per equation \ref{eq: MPC quadratic risk}.
\begin{equation}
    \psi_t(w_t) = w_t^T\Sigma_tw_t
    \label{eq: MPC quadratic risk}
\end{equation}

As such, the risk-adjustment charge $\psi_t(w_t)$ is proportional to the variance of the portfolio returns given the weights. $\Sigma_t$ represents an estimate of the return covariance, assuming that returns behave stochastically. Intuitively, this means that the term can be considered a cost that discourages portfolios with high variance. 

When combining the objective function in equation \ref{eq: MPC_stochastic} with the quadratic risk function of equation \ref{eq: MPC quadratic risk} it results in assuming that investors possess and act upon mean-variance preferences. Furthermore, if the returns are independent random variables, the objective will be equivalent to the mean-variance criterion introduced by Markowitz (1952)\footnote{
In which the investor objective function is $\max \left(E[r] - \frac{1}{2}\gamma Var[r] \right)$}. This thesis acknowledges the predominant use of quadratic risk due to its intutive nature as well as its straight-forwardness in terms of deriving a solution through convex optimization. Yet, there is an increasing focus on alternative risk measures beyond the introduced quadratic risk from equation \ref{eq: MPC quadratic risk}. Many of these alternative risk measures are also convex hence they are usable in the MPC framework. The alternative risk measures include expected shortfall, defined as the loss a portfolio manager could expect in the worst x\% of cases (Munk, 2019). It is a coherent risk measure encompassing the neat property that it only penalizes downside similar to the Sortino ratio. However, when portfolios are constructed to minimize expected shortfall, empirical studies, such as Lim et al. (2011), have shown that they often realize a higher expected shortfall out of sample when compared to mean variance efficient portfolios. The uncertainty is increasing in the lowering of the level x\%. 

To circumvent this, investors concerned with the risk of the left tail of the distribution, can use drawdown control functions as an appealing alternative to control risk, since it prevents a portfolio of losing more than a given pre-defined acceptable level. 

\subsubsection{Drawdown control}
A portfolio manager is often subject and constrained by a maximum drawdown level which means that at each point in time $t$ the portfolio can not drop below a fixed value. Remember that $V_t$ entails the value of the portfolio at time $t$. Furthermore, the maximum value of the portfolio in the past is defined as,
\begin{equation}
    M_t = \max_{\tau \leq t} V_t
\end{equation}
meaning that the drawdown at time $t$ is defined as 
\begin{equation}
    D_t = 1 - \frac{V_t}{M_t}
\end{equation}

Grossman \& Zhou (1993) were the first to research and study the nature of portfolio selection under the constraint that the portfolio value must never drop below a specific percentage of the rolling maximum portfolio value. Particularly, the researchers found that when dealing with utility functions exhibiting constant relative risk aversion, the optimal allocation to risky assets at time $t$ is in proportion to the cushion $D^{max}-D_t$, where $D^{max} \in (0,1)$ is the maximum drawdown that the portfolio manager can withstand. This can be implemented by adjusting the risk-aversion parameter as changes to the cushion occur (Nystrup, 2017).

Define the risk aversion parameter as $\gamma_0$ when $D_t = 0$, meaning that $V_t = M_t$. This means that $\gamma_0$ is the initial and minimal risk aversion since $V_0 = M_0$. Furthermore, it should be noted that the drawdown can never become negative, meaning that $M_t \geq V_t$. Furthermore, it intuitively follows that when $D_t = D^{max}$, the allocation to risky assets should be 0 resulting in an infinite risk aversion evident by equation \ref{eq: MPC risk aversion}.
\begin{equation}
\gamma_t = \gamma_0 \frac{D^{max}}{D^{max}-D_t}
    \label{eq: MPC risk aversion}
\end{equation}

In practise, the denominator in equation \ref{eq: MPC risk aversion} will be replaced by $max(D^{max}-D_t,\epsilon)$, in which $\epsilon$ is a small number in order to avoid dividing by 0 or negative numbers. Additionally, it should be noted that $\gamma_\tau$ is only adjusted based on the realized drawdown, thereby resulting in keeping $\gamma_\tau = \gamma_t$ for $\tau = t+1,...., t+H$ when solving the maximizing problem in equation \ref{eq: maximizing objective MPC}.

Conclusively, the formulation of the risk-aversion coefficient in equation \ref{eq: MPC risk aversion} means that risk-aversion will increase when the portfolio encounters losses, thereby implying a path-dependent utility function. One of the main critiques against this type of utility function is that if the drawdown gets too close to the predefined limit it can be impossible to escape a high risk-aversion coefficient meaning that the portfolio manager is unable to efficiently allocate capital. This implies that the lower the maximum drawdown limit, $D^{max}$ is set, the larger the risk of being stuck at the limit. In practise this means that the portfolio manager, who gets trapped at the limit, must either go to the client or the investment committee to ask for a new limit.

\subsubsection{Transaction costs}
One of the most important considerations when comparing the performance of static and dynamic portfolio strategies evolves around transaction costs as frequent trading can offset the potential benefits of the dynamic strategy. As such, trading and the associated transaction costs, should be regularized in order to reduce the risk of trading too frequently. However, the regularization should not be too constraining such that portfolio managers have difficulties capitalizing on evident capital allocation opportunities. The penalty for trading that is to be included in the objective function of equation \ref{eq: maximizing objective MPC} can be defined as, 

\begin{equation}
    \phi_t^{trade}(w_t-w_{t-1}) = \kappa_1^T|w_t-w_{t-1}| + \kappa_2^T(w_t-w_{t-1})^2
    \label{eq: MPC trade function}
\end{equation}

in which $\kappa_1$ and $\kappa_2$ are vectors of penalty factors represented by the absolute and squared value elementwise. From an intuitive standpoint this trading function could reflect the actual transaction costs or a restrictive attitude towards trading due to the uncertainty associated with forecasting from section \ref{section: forecasting MPC HMM}. The penalty function represented in equation \ref{eq: MPC trade function} is a weighted elastic net, thereby serving as a combination of the ridge and lasso penalizers. This means that the penalty function is a convex combination of the $l_1$ and the squared $l_2$ norm. Intuitively, the $l_1$ penalty reduces the number of trades and the $l_2$ penalty reduces the size of the trades. More precisely, the $l_2$ penalty shrinks together trades in highly correlated assets and splits the trades over multiple days (Nystrup, 2017). As such, the $l_2$ penalty tries to circumvent the price impact of large orders since these large orders tend to move market prices in the assets that are traded. 

It is possible to incorporate a variety of different trading penalty functions, for instance Gârleanu \& Pedersen (2013) presented a cost function of the nature $(w_t-w_{t-1})^T\Sigma_t(w_t-w_{t-1})$ which is closely related to the quadratic risk function of equation \ref{eq: MPC quadratic risk}, meaning that it captures the increased cost of trading in markets with high volatility.

Despite this alternative, the intuitive nature of the $l_1$ and $l_2$ penalties of equation \ref{eq: MPC trade function} and their direct link to linear regression regularization through the LASSO, Ridge and Elastic net methodologies, makes it the preferred choice for the subsequent analysis.

\subsubsection{Holding costs}
Another crucial consideration for portfolio managers concerns the fact that holding a particular portfolio $w_t$ over $t$ periods can result in a holding cost. An intutive and basic holding function can be constructed in a similar fashion to the aforementioned trading penalty in equation \ref{eq: MPC trade function},

\begin{equation}
    \phi_t^{hold}(w_t) = p_1^T|w_t| + p_2^Tw_t^2
    \label{eq: MPC holding function}
\end{equation}

where $p_1$ and $p_2$ are penalty vectors representing the absolute and squared penalty value elementwise. This means that the portfolio manager will be restricted to a long-only capital allocation strategy when the holding costs grow large enough. As such, the intuitive reason for including a holding costs penalty function is to control the leverage of the portfolio. Additionally, the theoretical foundation for the definition of equation \ref{eq: MPC holding function} is identical to the outlined reasoning for the trading function of equation \ref{eq: MPC trade function}. The $l_1$ penalty reduces the number of different holdings while the $l_2$ penalty reduces the size of the different holdings.

\subsubsection{Constraints}
In addition to controlling trading and holding costs, portfolio managers can potentially improve the out-of-sample performance by imposing constraints on the portfolio allocation, which is essentially identical to shrinking the covariance matrix (Jagannathan \& Ma, 2003). As such, the portfolio manager can impose constraints on the weights of the different assets such as minimum and maximum positions,
\begin{equation}
    -w^{min}\leq w_t\leq w^{max},
\end{equation}
in which the inequalities are elementwise and $w^{min}$ and $w^{max}$ represent non-negative vectors of the allowed maximum short and long positions in a given asset respectively. In this scenario, a long-only portfolio will correspond to $w^{min}=0$.

Finally, a maximum leverage constraint is implemented as,
\begin{equation}
    || (w_t)_{1:n} ||_1 \leq L^{max}
\end{equation}

which requires the leverage in risky assets to not exceed $L^{max}$. For long-short and leveraged long-only the thesis impose the constraint $L^{max}=2$ going forward which is in line with that used by Nystrup (2017). However, it should be noted that the leverage constraint is a hyper-parameter that can be subject to additional tuning. Other direct constraint such as asset specific constraints can be imposed, however, such possible constraints will be mentioned in the results section.


\textbf{Write a short section on forecast error risk and how this is regularized in this framework.}

In conclusion, the section has introduced and elaborated on the functionality of the MPC framework as well as some of the different choices of cost functions to be included in the objective function defined in equation \ref{eq: maximizing objective MPC}. It should be noted that the above findings and arguments are selective examples based on a wide variety of possible convex holding and trading costs functions that could have been implemented just as easily. For further elaboration of such the author refer to the study conducted by Boyd et al. (2017) in which many other convex holding and trading costs functions are introduced. As such, the thesis has introduced the two estimators in section $\ref{section: estimation}$ after which the estimators were validated by showcasing that both models could somewhat reproduce the stylized facts introduced by Granger \& Ding (1995b) in section \ref{Section: Stylized facts}. Furthermore, the current section has introduced the MPC framework which will be used as part of the portfolio exercise going forward. As such, the thesis stands at a point where the remaining analysis to be conducted involves estimating the results of combining the estimators with the portfolio framework to evaluate whether superior performance in terms of a variety of different return metrics can be achieved. This means that the coming section will focus on the generated results. 