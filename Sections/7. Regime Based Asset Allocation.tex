\newpage

\section{Regime Based Asset Allocation}
Portfolio allocation boils down to essentially maximizing the return generated for a given amount of risk. The portfolio manager who successfully and consistently does so, will have more money to manage and more satisfied investors. Generally speaking there are three approaches when it comes to portfolio selection that sophisticated institutional investors utilize (Nystrup, 2017). The first approach consists of diversification, which entails increasing risk-adjusted returns by forming optimal portfolios on the basis of allocating capital towards imperfectly correlated assets. This is also known as static asset allocation, since the portfolio allocation is constructed to capitalize on bullish market periods while providing protection in bearish market periods, hence static asset allocation is considered an all-weather portfolio (Markowitz, 1952). However, the challenge with static all-weather portfolios is that in order for them to remain efficient they must be continuously re-balanced and this trading is subject to trading costs, thereby negatively impacting returns. Furthermore, and perhaps more importantly, the financial crisis of 2008 demonstrated that diversification is not sufficient to avoid large drawdowns (Nystrup, 2017). As such, diversification fails when investors need it the most since the correlations between risky assets have a tendency to strengthen when markets are characterised by high volatility (Pedersen, 2009).  


Short intro to regime based asset allocation - evt. sammenlign med statisk aktiv allokering og aktiv allokering mod statiske vægte

\subsection{Model Predictive cotrol}



Introducer modellen
- Object-funcktion + constriant.

Underafsnit hvor man går mere i dybden med valget og udregningen af nogle dele af objekt-funktionen.

\subsection{Data...}


\subsection{Results}




\subsubsection{Comparison}
